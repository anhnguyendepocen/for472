\documentclass[]{krantz}
\usepackage{lmodern}
\usepackage{amssymb,amsmath}
\usepackage{ifxetex,ifluatex}
\usepackage{fixltx2e} % provides \textsubscript
\ifnum 0\ifxetex 1\fi\ifluatex 1\fi=0 % if pdftex
  \usepackage[T1]{fontenc}
  \usepackage[utf8]{inputenc}
\else % if luatex or xelatex
  \ifxetex
    \usepackage{mathspec}
  \else
    \usepackage{fontspec}
  \fi
  \defaultfontfeatures{Ligatures=TeX,Scale=MatchLowercase}
\fi
% use upquote if available, for straight quotes in verbatim environments
\IfFileExists{upquote.sty}{\usepackage{upquote}}{}
% use microtype if available
\IfFileExists{microtype.sty}{%
\usepackage[]{microtype}
\UseMicrotypeSet[protrusion]{basicmath} % disable protrusion for tt fonts
}{}
\PassOptionsToPackage{hyphens}{url} % url is loaded by hyperref
\usepackage[unicode=true]{hyperref}
\PassOptionsToPackage{usenames,dvipsnames}{color} % color is loaded by hyperref
\hypersetup{
            pdftitle={Forestry 472: Ecological Monitoring and Data Analysis},
            pdfauthor={Andrew O. Finley and Jeffrey W. Doser},
            colorlinks=true,
            linkcolor=Maroon,
            citecolor=Blue,
            urlcolor=Blue,
            breaklinks=true}
\urlstyle{same}  % don't use monospace font for urls
\usepackage{natbib}
\bibliographystyle{apalike}
\usepackage{longtable,booktabs}
% Fix footnotes in tables (requires footnote package)
\IfFileExists{footnote.sty}{\usepackage{footnote}\makesavenoteenv{long table}}{}
\usepackage{graphicx,grffile}
\makeatletter
\def\maxwidth{\ifdim\Gin@nat@width>\linewidth\linewidth\else\Gin@nat@width\fi}
\def\maxheight{\ifdim\Gin@nat@height>\textheight\textheight\else\Gin@nat@height\fi}
\makeatother
% Scale images if necessary, so that they will not overflow the page
% margins by default, and it is still possible to overwrite the defaults
% using explicit options in \includegraphics[width, height, ...]{}
\setkeys{Gin}{width=\maxwidth,height=\maxheight,keepaspectratio}
\IfFileExists{parskip.sty}{%
\usepackage{parskip}
}{% else
\setlength{\parindent}{0pt}
\setlength{\parskip}{6pt plus 2pt minus 1pt}
}
\setlength{\emergencystretch}{3em}  % prevent overfull lines
\providecommand{\tightlist}{%
  \setlength{\itemsep}{0pt}\setlength{\parskip}{0pt}}
\setcounter{secnumdepth}{5}
% Redefines (sub)paragraphs to behave more like sections
\ifx\paragraph\undefined\else
\let\oldparagraph\paragraph
\renewcommand{\paragraph}[1]{\oldparagraph{#1}\mbox{}}
\fi
\ifx\subparagraph\undefined\else
\let\oldsubparagraph\subparagraph
\renewcommand{\subparagraph}[1]{\oldsubparagraph{#1}\mbox{}}
\fi

% set default figure placement to htbp
\makeatletter
\def\fps@figure{htbp}
\makeatother

\usepackage{booktabs}
\usepackage{longtable}
\usepackage[bf,singlelinecheck=off]{caption}

\usepackage{framed,color}
\definecolor{shadecolor}{RGB}{248,248,248}

\renewcommand{\textfraction}{0.05}
\renewcommand{\topfraction}{0.8}
\renewcommand{\bottomfraction}{0.8}
\renewcommand{\floatpagefraction}{0.75}

\renewenvironment{quote}{\begin{VF}}{\end{VF}}
\let\oldhref\href
\renewcommand{\href}[2]{#2\footnote{\url{#1}}}

\makeatletter
\newenvironment{kframe}{%
\medskip{}
\setlength{\fboxsep}{.8em}
 \def\at@end@of@kframe{}%
 \ifinner\ifhmode%
  \def\at@end@of@kframe{\end{minipage}}%
  \begin{minipage}{\columnwidth}%
 \fi\fi%
 \def\FrameCommand##1{\hskip\@totalleftmargin \hskip-\fboxsep
 \colorbox{shadecolor}{##1}\hskip-\fboxsep
     % There is no \\@totalrightmargin, so:
     \hskip-\linewidth \hskip-\@totalleftmargin \hskip\columnwidth}%
 \MakeFramed {\advance\hsize-\width
   \@totalleftmargin\z@ \linewidth\hsize
   \@setminipage}}%
 {\par\unskip\endMakeFramed%
 \at@end@of@kframe}
\makeatother

\renewenvironment{Shaded}{\begin{kframe}}{\end{kframe}}

\usepackage{makeidx}
\makeindex

\urlstyle{tt}

\usepackage{amsthm}
\makeatletter
\def\thm@space@setup{%
  \thm@preskip=8pt plus 2pt minus 4pt
  \thm@postskip=\thm@preskip
}
\makeatother

\frontmatter

\title{Forestry 472: Ecological Monitoring and Data Analysis}
\author{Andrew O. Finley and Jeffrey W. Doser}
\date{2018-08-30}

\usepackage{amsthm}
\newtheorem{theorem}{Theorem}[chapter]
\newtheorem{lemma}{Lemma}[chapter]
\theoremstyle{definition}
\newtheorem{definition}{Definition}[chapter]
\newtheorem{corollary}{Corollary}[chapter]
\newtheorem{proposition}{Proposition}[chapter]
\theoremstyle{definition}
\newtheorem{example}{Example}[chapter]
\theoremstyle{definition}
\newtheorem{exercise}{Exercise}[chapter]
\theoremstyle{remark}
\newtheorem*{remark}{Remark}
\newtheorem*{solution}{Solution}
\begin{document}
\maketitle

% you may need to leave a few empty pages before the dedication page

%\cleardoublepage\newpage\thispagestyle{empty}\null
%\cleardoublepage\newpage\thispagestyle{empty}\null
%\cleardoublepage\newpage
\thispagestyle{empty}

\begin{center}
To my son,

without whom I should have finished this book two years earlier
%\includegraphics{images/dedication.pdf}
\end{center}

\setlength{\abovedisplayskip}{-5pt}
\setlength{\abovedisplayshortskip}{-5pt}

{
\hypersetup{linkcolor=black}
\setcounter{tocdepth}{2}
\tableofcontents
}
\listoftables
\listoffigures
\chapter*{Preface}\label{preface}


This text is an introduction to data sciences for Forestry and
Environmental students. Understanding and responding to current
environmental challenges requires strong quantitative and analytical
skills. There is a pressing need for professionals with data science
expertise in this data rich era. The
\href{http://www.mckinsey.com/insights/business_technology/big_data_the_next_frontier_for_innovation}{McKinsey
Global Institute} predicts that ``by 2018, the United States alone could
face a shortage of 140,000 to 190,000 people with deep analytical skills
as well as 1.5 million managers and analysts with the know-how to use
the analysis of big data to make effective decisions''. The Harvard
Business Review dubbed \emph{data scientist}
\href{https://hbr.org/2012/10/data-scientist-the-sexiest-job-of-the-21st-century}{``The
Sexiest Job of the 21st Century''}. This need is not at all confined to
the tech sector, as forestry professionals are increasingly asked to
assume the role of \emph{data scientists} and \emph{data analysts} given
the rapid accumulation and availability of environmental data (see, e.g.
\citet{Schimel2015}).
\href{www.import.io/post/data-scientists-vs-data-analysts-why-the-distinction-matters}{Thomson
Nguyen's talk} on the difference between a data scientist and a data
analyst is very interesting and contains elements relevant to the aim of
this text. This aim is to give you the opportunity to acquire the tools
needed to become an environmental data analyst. Following
\citet{Bravo16} a \emph{data analyst} has the ability to make
appropriate calculations, convert data to graphical representation,
interpret the information presented in graphical or mathematical forms,
and make judgements or draw conclusions based on the quantitative
analysis of data.

\chapter{Data}\label{data}

\section{FEF Tree Biomass Data Set}\label{fef-tree-biomass-data-set}

When thinking about data, we might initially have in mind a modest-sized
and uncomplicated data set that serves a fairly specific purpose. For
example, in forestry it is convenient to have a mathematical formula
that relates a tree's diameter (or some other easily measured attribute)
to stem or total biomass (i.e.~we cannot directly measure tree biomass
without destructive sampling). When coupled with forest inventory data,
such formulas provide a means to estimate forest biomass across
management units or entire forest landscapes. A data set used to create
such formulas includes felled tree biomass by tree component for four
hardwood species of the central Appalachians sampled on the
\href{http://www.nrs.fs.fed.us/ef/locations/wv/fernow}{Fernow
Experimental Forest} (FEF), West Virginia \citet{Wood2016}. A total of
88 trees were sampled from plots within two different watersheds on the
FEF. Hardwood species sampled include \emph{Acer rubrum}, \emph{Betula
lenta}, \emph{Liriodendron tulipifera}, and \emph{Prunus serotina}, all
of which were measured in teh summer of 1991 and 1992. Data include tree
height, diameter, as well as green and dry weight of tree stem, top,
small branches, large branches, and leaves.

\bibliography{text.bib}

\backmatter
\printindex

\end{document}
